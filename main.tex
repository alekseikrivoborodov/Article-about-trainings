\documentclass{article}

% Language setting
% Replace `english' with e.g. `spanish' to change the document language
\usepackage[russian]{babel}

% Set page size and margins
% Replace `letterpaper' with `a4paper' for UK/EU standard size
\usepackage[letterpaper,top=2cm,bottom=2cm,left=3cm,right=3cm,marginparwidth=1.75cm]{geometry}

% Useful packages
\usepackage{amsmath}
\usepackage{graphicx}
\usepackage[colorlinks=true, allcolors=blue]{hyperref}

\begin{figure}
\includegraphics[width=1\textwidth]{almau.png}
\end{figure}

\title{\textbf{Структурный обзор тренингов, маркетинговой модели и аудитории, которая ими интересуется.}}
\raggedright\author{Кривобородов Алексей \\ AlmaU \\ ИС 4 курс рус \\ Казахстан, Алматы \\ \textbf{Research-Methods}}

\begin{document}
\maketitle
\newpage
\tableofcontents


\section{Цели:}
\begin{itemize} 
\item Определить понятия и типы тренингов
\item Определить их маркетинговую модель
\item Определить их целевую и конечную аудитории
\item Подвести итог о современной модели тренингов
\end{itemize}

\section{Актуальность:}
По данным портала Informs - аудитория онлайн тренингов только на русском языке уже превысила 3.000.000 купивших курсы. Это только задекламированная статистика, если учесть бесплатные сливы курсов, частные и небольшие онлайн-академии, и просто тех кто не вошел в подсчет это цифра может быть в несколько раз больше. И также по данным портала Informs эта цифра растет в геометрической прогрессии. \\
\textbf{Тема актуальна.}

\newpage

\section{Аннотации:}
\begin{itemize} 
\item “прогрев” - действия применяемые, для того чтобы создать у аудитории интерес к продукту
\item “инфлюенсер” - человек, с аудиторией в какой-то сети, локальный лидер мнений
\item “блогер” - человек с аудиторией в сети, существующей вокруг него
\item “мусорность”, “бесполезность” - неактуальность
\item “вода” - материал не по теме, размусоливание
\item “бизнес-тренинг” - тренинг направленный на бизнес знания, или личную эффективность
\end{itemize}


\section{Сводка:}
Для статьи использовались следующие материалы:
\newline
\begin{enumerate}
    \item \textbf{GetCourse} - \textit{“методичка создателя тренингов”}
    \item \textbf{Анна Торшерская} - \textit{“О создании правильных курсов”}
    \item \textbf{Informs} - \textit{портал со статистикой в разных сферах маркетинга}
    \item \textbf{Atmosphere } - \textit{AlfaCRM статистика}
\end{enumerate}
Также статья построена на личном обзоре большого числа тренингов в каждой сфере.

\section{Общая информация:}
\paragraph{}
Онлайн школы. Онлайн академии. Курсы. Мастер-классы. Тренинги. Все эти слова заполнили современные мультимедийные каналы связи. В современном медиа поле тяжело проводить больше часа в сети, и ни разу не встретить рекламы какой-нибудь онлайн школы, которая учит чему-то невероятному. 
\paragraph{}
На самом деле все это разные понятия и что-то из этого заслуживает внимания и своего покупателя. В этой статье для того чтобы не размывать понятия, и растягивать повествования мы сойдемся на мысли, что все мастер-классы, курсы, семинары являются просто одним из видов тренинга, которые уже будем делить на группы, и разбирать их целевые аудитории. 
\paragraph{}
В целом можно разделить все виды онлайн курсов на следующие категории:
\begin{itemize}
    \item мотивационные тренинги
    \item тематические тренинги
    \item коуч-тренинги
    \item мастер-класс тренинги
    \item тренинги повышения квалификации
    \item тренинги смены квалификации
\end{itemize}

\newpage

\section{Тренинги:}
\subsection{Мотивационные тренинги:}
\paragraph{\textbf{Описание:}}
\paragraph{}
Мотивационные тренинги - примитивная мотивация существующая только для заработка на человеческой грусти и не определенности. Почти каждый бизнес-тренинг в конечном итоге превращается в мотивационный, в котором учащемуся каждый 10 минут напоминают о том, что у него все получится, и заряжают человека мотивацией. Мотивационные тренинги чаще всего проводятся блогерами, инфлюенсерами и подобными им. Монетизируются засчет аудитории фанатов, и являются самым низкосортным продуктом среди всех типов тренингом.
\paragraph{\textbf{Плюсы:}}
\begin{itemize}
    \item Кратковременная мотивация на работу
\end{itemize}
\paragraph{\textbf{Минусы:}}
\begin{itemize}
    \item Бесполезные знания
    \item Мотивация значительно слабее, чем дисциплина
    \item Стоимость курсов неоправданно завышена, из-за медийности персоны
\end{itemize}
\paragraph{\textbf{Кто интересуется?}}
Фанаты, люди которые способны превращать обычного человека в идола и поглощать все то, что он производит. Также есть специальная аудитория людей, которая массово и бездумно занимаются мусорным самообразованием. Эти люди скупают и проходят курсы от медийных личностей, в надежде увидеть тот самый ответ, который расскажет как правильно жить эту жизнь. 


\subsection{Тематические тренинги:}
\paragraph{\textbf{Описание:}}
\paragraph{}
Тематические тренинги - тренинги изначально сделанные для узкой категории граждан. Примеры современные тренинги изменения сознания, тренинги на которых учат мечтать, и прочий мусор собранный на коленке, для того заработать на человеческой глупости. Безусловно, тематический тренинг может быть интересным, и нужным, когда его проводит палеонтолог, для археологов, что рассказать им новые исследования, гипотезы, и т.д. Когда тренинг выполняет свое прямое назначение и люди с узконаправленными знан
\paragraph{\textbf{Плюсы:}}
\begin{itemize}
    \item Если тематические тренинги выполняют свою задачу, это очень эффективный способ поднятия квалификации
    \item Узконаправленность темы объединяет людей
\end{itemize}
\paragraph{\textbf{Минусы:}}
\begin{itemize}
    \item “Инфоцыгане” проводят их чаще других
    \item Потенциальная бесполезность знаний
    \item Разбор выдуманных или неактуальных проблем
\end{itemize}
\paragraph{\textbf{Кто интересуется?}}
Узкая аудитория для которой был и подготовлен весь тренинг. Или как чаще получается в интернете, все те, кому промыли голову рекламой, и мнениями о том, как правильно мечтать, изменять свое сознание, смотреть на мир, и чистить лук. По своей сути является самым неоднозначным тренингом, с одной стороны нужный, с другой стороны захвачен “инфоцыганами”.


\subsection{Коуч тренинги:}
\paragraph{\textbf{Описание:}}
\paragraph{}
Коуч-тренинги - тренинги построены на личности коуча, и подвязанные под какую-то сферу саморазвития, почти всегда бизнес-мышление, тайм-менеджмент, личная эффективность, и прочие подарки капитализма. Также есть яркие примеры хороших работ, такие как тренинги Фридмана “По менеджменту и управлению персонала”. Где опытный человек, рассказывает об узкой части строения бизнеса, но из-за избыточной рекламы и замыленности глаза людей в интернете, такие продукты остаются малоизвестными. Ведь они узконаправленные, и требуют большой работы. Нет, популярность в этом сегменте имеют тренинги а-ля за один курс узнай все что нужно, чтобы построить бизнес, семинар на час об инвестициях.
\paragraph{\textbf{Плюсы:}}
\begin{itemize}
    \item Являются очень полезными, когда их проводит яркий специалист и оратор в своей сфере.
\end{itemize}
\paragraph{\textbf{Минусы:}}
\begin{itemize}
    \item “Инфоцыгане” проводят их чаще других
    \item Крайне тяжело найти полезные
    \item Завышенная цена, ввиду медийности персоны, что их ведет
    \item Постоянные “прогревы в сети”
\end{itemize}
\paragraph{\textbf{Кто интересуется?}}
Узкая аудитория для которой был и подготовлен качественный тренинг. В случае с Фридманом, например руководители и топ-менеджмент. Но чаще всего аудитория широкая и тренингами интересуются просто те, кого зацепит рекламный материал, рассказывающий об успех коуча, в той или иной сфере деятельности. 


\subsection{Мастер-класс тренинги:}
\paragraph{\textbf{Описание:}}
\paragraph{}
Мастер-класс тренинги(раньше семинары) - еще один прекрасный пример узконаправленного и качественного продукта. Делается изначально под определенную аудиторию, с базовым набором знаний в сфере. Например, мастер-класс для бухгалтеров для работы с новым Excel 2022, или мастер-класс работы с 3D обьектами в Photoshop. Изначально проводились вживую, сейчас все чаще проходят в Zoom и на других онлайн площадках. 
\paragraph{\textbf{Плюсы:}}
\begin{itemize}
    \item Являются очень полезными, когда их проводит яркий специалист и оратор в своей сфере.
    \item Сертификаты могут помочь в работе, и в общем зачёте квалификации.
    \item Объединяют людей.
\end{itemize}
\paragraph{\textbf{Минусы:}}
\begin{itemize}
    \item Часто имеют низкий бюджет, и являются довольно скучным и плохо поданным материалом.
\end{itemize}
\paragraph{\textbf{Кто интересуется?}}
Узкая аудитория для которой был и подготовлен тренинг. Люди работающей в той специальности, для которой и был создан тренинг, при том те, у кого есть время и желание дополнительно развиваться. В больших кампаниях, руководители сами отправляют сотрудников на подобные семинары, поэтому кампании чаще работают в режиме B2B, чем B2C.


\subsection{Тренинги повышения квалификации:}
\paragraph{\textbf{Описание:}}
\paragraph{}
Тренинги повышения квалификации - более углубленная версия мастер-класса, по сути является конечным продуктом. А мастер-классы используются как “прогрев” перед ними. И здесь уже теряется однозначная релевантность продукта. Мастер-класс - законченный продукт нацеленный на рекламу, в то время как тренинг повышения квалификации может быть сделан хуже, ведь его первичная цель продать. 
\paragraph{\textbf{Плюсы:}}
\begin{itemize}
    \item Являются очень полезными, когда их проводит яркий специалист и оратор в своей сфере.
    \item Сертификаты могут помочь в работе, и в общем зачёте квалификации.
\end{itemize}
\paragraph{\textbf{Минусы:}}
\begin{itemize}
    \item Часто могут оказаться пустой тратой времени, и “водой” не по теме.
    \item Стоят очень дорого, и часто своей цены не оправдывают.
\end{itemize}
\paragraph{\textbf{Кто интересуется?}}
Узкая аудитория прогретая мастер-классами. Узкая и широкая часть специалистов в области, подобные тренинги чуть ли не единственный тип, который не нуждается в широкой огласке в сети, и постоянной рекламе и прогревах, способен существовать на мастер-классах и личных рекомендациях. Также часто работает по системе B2B.

\subsection{Тренинги смены квалификации:}
\paragraph{\textbf{Описание:}}
\paragraph{}
Тренинг смены квалификации - самая быстро растущий и известный тип тренингов в сети. Яркие представители, школы GeekBrains, SkillBox, и прочие, маркетинг подобных школ настроен на людей недовольных своим заработком и своей профессией, коих в современном мире капитализма и постоянной рекламы бесконечно много. Курс позиционируется в погружении в профессию. Полной или частичной смене деятельности. 
\paragraph{\textbf{Плюсы:}}
\begin{itemize}
    \item Являются полезными, когда качественно сделаны и поданы.
    \item Являются самым качественным продуктом в индустрии впринципе.
\end{itemize}
\paragraph{\textbf{Минусы:}}
\begin{itemize}
    \item Быстро теряют актуальность
    \item Часто являются лишь начальными знаниями, и бегом по вершкам. 
    \item Имеют очень завышенную цену из-за большой маркетинговой системы
    \item Часто, не оправдывают ожиданий
\end{itemize}
\paragraph{\textbf{Кто интересуется?}}
Широкая аудитория привлеченная таргетированной рекламой, различная аудитория привлеченная блогерами, инфлюинсерами. Единственный тип тренингов, где не работает сарафонное радио, ввиду передачи аккаунтов. 




\section{Резюме:}
\begin{quote}
    Тренинги прекрасное достижение онлайн образования, уничтоженное человеческой глупостью, жадностью, некачественными продуктами и перенасыщенностью рекламой в сети. Тренинги можно разделить на рассчитанные на широкую аудиторию, и на рассчитанную на узкую категорию специалистов. Интересуется тренингами аудитория 14 - 40 лет. Разбор и сегментация аудитории напрямую зависит от материала. Маркетинговая система выстроена вокруг таргетированной рекламы и сарафанного радио. 
\end{quote}

\section{Summary:}
\begin{quote}
    Coaching is a wonderful achievement of online education, destroyed by human stupidity, greed, substandard products and oversaturation of advertising on the web. Trainings can be divided into those designed for a wide audience, and those designed for a narrow category of specialists. Interested in training audience 14 - 40 years. Analysis and segmentation of the audience directly depends on the material. The marketing system is built around targeted advertising and word of mouth. 
\end{quote}

\section*{Таблица:}
\begin{table}[ht]
% \caption{}
% \label{tab:my-table}
\begin{tabular}{|ll|l|ll}
\cline{1-3}
\multicolumn{2}{|l|}{Материалы} & Количество &  &  \\ \cline{1-3}
\multicolumn{1}{|l|}{\(\displaystyle A =\pi r^{2} \) }   & {\(\displaystyle \pi  \) }    & 178        &  &  \\ \cline{1-3}
\multicolumn{1}{|l|}{\(\alpha \) }    & {\(\displaystyle \sqrt[5]{a} + \cos(x)) + \lim_{n\rightarrow \infty } (1 + \frac{1}{n})^{n} \) }    & 45         &  &  \\ \cline{1-3}
\end{tabular}
\end{table}


\begin{figure}[ht]
\includegraphics[width=1\textwidth]{divider.png}
\end{figure}

\end{document}